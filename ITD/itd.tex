% Options for packages loaded elsewhere
\PassOptionsToPackage{unicode}{hyperref}
\PassOptionsToPackage{hyphens}{url}
%
\documentclass[
]{article}
\usepackage{amsmath,amssymb}
\usepackage{lmodern}
\usepackage{ifxetex,ifluatex}
\ifnum 0\ifxetex 1\fi\ifluatex 1\fi=0 % if pdftex
  \usepackage[T1]{fontenc}
  \usepackage[utf8]{inputenc}
  \usepackage{textcomp} % provide euro and other symbols
\else % if luatex or xetex
  \usepackage{unicode-math}
  \defaultfontfeatures{Scale=MatchLowercase}
  \defaultfontfeatures[\rmfamily]{Ligatures=TeX,Scale=1}
\fi
% Use upquote if available, for straight quotes in verbatim environments
\IfFileExists{upquote.sty}{\usepackage{upquote}}{}
\IfFileExists{microtype.sty}{% use microtype if available
  \usepackage[]{microtype}
  \UseMicrotypeSet[protrusion]{basicmath} % disable protrusion for tt fonts
}{}
\makeatletter
\@ifundefined{KOMAClassName}{% if non-KOMA class
  \IfFileExists{parskip.sty}{%
    \usepackage{parskip}
  }{% else
    \setlength{\parindent}{0pt}
    \setlength{\parskip}{6pt plus 2pt minus 1pt}}
}{% if KOMA class
  \KOMAoptions{parskip=half}}
\makeatother
\usepackage{xcolor}
\IfFileExists{xurl.sty}{\usepackage{xurl}}{} % add URL line breaks if available
\IfFileExists{bookmark.sty}{\usepackage{bookmark}}{\usepackage{hyperref}}
\hypersetup{
  hidelinks,
  pdfcreator={LaTeX via pandoc}}
\urlstyle{same} % disable monospaced font for URLs
\usepackage{color}
\usepackage{fancyvrb}

%--------------------- OUR COMMANDS ---------------------------%
\usepackage[scaled]{helvet}
\renewcommand\familydefault{\sfdefault} 
\usepackage[T1]{fontenc}
\usepackage[english]{babel}
\usepackage[margin=1in]{geometry}
\usepackage{graphicx}
%--------------------------------------------------------------%

\newcommand{\VerbBar}{|}
\newcommand{\VERB}{\Verb[commandchars=\\\{\}]}
\DefineVerbatimEnvironment{Highlighting}{Verbatim}{commandchars=\\\{\}}
% Add ',fontsize=\small' for more characters per line
\newenvironment{Shaded}{}{}
\newcommand{\AlertTok}[1]{\textcolor[rgb]{1.00,0.00,0.00}{\textbf{#1}}}
\newcommand{\AnnotationTok}[1]{\textcolor[rgb]{0.38,0.63,0.69}{\textbf{\textit{#1}}}}
\newcommand{\AttributeTok}[1]{\textcolor[rgb]{0.49,0.56,0.16}{#1}}
\newcommand{\BaseNTok}[1]{\textcolor[rgb]{0.25,0.63,0.44}{#1}}
\newcommand{\BuiltInTok}[1]{#1}
\newcommand{\CharTok}[1]{\textcolor[rgb]{0.25,0.44,0.63}{#1}}
\newcommand{\CommentTok}[1]{\textcolor[rgb]{0.38,0.63,0.69}{\textit{#1}}}
\newcommand{\CommentVarTok}[1]{\textcolor[rgb]{0.38,0.63,0.69}{\textbf{\textit{#1}}}}
\newcommand{\ConstantTok}[1]{\textcolor[rgb]{0.53,0.00,0.00}{#1}}
\newcommand{\ControlFlowTok}[1]{\textcolor[rgb]{0.00,0.44,0.13}{\textbf{#1}}}
\newcommand{\DataTypeTok}[1]{\textcolor[rgb]{0.56,0.13,0.00}{#1}}
\newcommand{\DecValTok}[1]{\textcolor[rgb]{0.25,0.63,0.44}{#1}}
\newcommand{\DocumentationTok}[1]{\textcolor[rgb]{0.73,0.13,0.13}{\textit{#1}}}
\newcommand{\ErrorTok}[1]{\textcolor[rgb]{1.00,0.00,0.00}{\textbf{#1}}}
\newcommand{\ExtensionTok}[1]{#1}
\newcommand{\FloatTok}[1]{\textcolor[rgb]{0.25,0.63,0.44}{#1}}
\newcommand{\FunctionTok}[1]{\textcolor[rgb]{0.02,0.16,0.49}{#1}}
\newcommand{\ImportTok}[1]{#1}
\newcommand{\InformationTok}[1]{\textcolor[rgb]{0.38,0.63,0.69}{\textbf{\textit{#1}}}}
\newcommand{\KeywordTok}[1]{\textcolor[rgb]{0.00,0.44,0.13}{\textbf{#1}}}
\newcommand{\NormalTok}[1]{#1}
\newcommand{\OperatorTok}[1]{\textcolor[rgb]{0.40,0.40,0.40}{#1}}
\newcommand{\OtherTok}[1]{\textcolor[rgb]{0.00,0.44,0.13}{#1}}
\newcommand{\PreprocessorTok}[1]{\textcolor[rgb]{0.74,0.48,0.00}{#1}}
\newcommand{\RegionMarkerTok}[1]{#1}
\newcommand{\SpecialCharTok}[1]{\textcolor[rgb]{0.25,0.44,0.63}{#1}}
\newcommand{\SpecialStringTok}[1]{\textcolor[rgb]{0.73,0.40,0.53}{#1}}
\newcommand{\StringTok}[1]{\textcolor[rgb]{0.25,0.44,0.63}{#1}}
\newcommand{\VariableTok}[1]{\textcolor[rgb]{0.10,0.09,0.49}{#1}}
\newcommand{\VerbatimStringTok}[1]{\textcolor[rgb]{0.25,0.44,0.63}{#1}}
\newcommand{\WarningTok}[1]{\textcolor[rgb]{0.38,0.63,0.69}{\textbf{\textit{#1}}}}
\usepackage[normalem]{ulem}
% Avoid problems with \sout in headers with hyperref
\pdfstringdefDisableCommands{\renewcommand{\sout}{}}
\setlength{\emergencystretch}{3em} % prevent overfull lines
\providecommand{\tightlist}{%
  \setlength{\itemsep}{0pt}\setlength{\parskip}{0pt}}
% \setcounter{secnumdepth}{-\maxdimen} % remove section numbering
\ifluatex
  \usepackage{selnolig}  % disable illegal ligatures
\fi

%--------------- DOCUMENT START -----------------%

\begin{document}

\begin{titlepage}
	\centering
    
    {\normalsize 
        Software Engineering 2 - Prof. Di Nitto Elisabetta \\ 
		Dipartimento di Elettronica, Informazione e Bioingegneria \\
        Politecnico di Milano \par
    }     \vspace{3cm}

    {\Huge \textbf{CLup - Customer Line-up\\} }    \vspace{1cm}
  
    {\large \textbf{ITD\\Implementation and Testing Delieverable} \par}     \vspace{4cm}

	{\normalsize Andrea Franchini(10560276) \\ Ian Di Dio Lavore (10580652)\\ Luigi Fusco(10601210)\par}     \vspace{3cm}

    \includegraphics[scale=0.4]{../DD/images/logo.pdf}
    \vspace{0.5cm}


	{\normalsize 10-01-2020 \par}
	
\end{titlepage}

\tableofcontents

%------------------ CONTENT --------------------%

\hypertarget{introduction}{%
\section{Introduction}\label{introduction}}

\hypertarget{scope}{%
\subsection{Scope}\label{scope}}

This document describes the implementation and testing process of a
working prototype of the service described in the ``Requirement Analysis
and Specification Document'' and ''Design Document''. This document is
intended to be a reference for the developer team and explains the
choices regarding used software, frameworks, programming languages. It
also provide input on how to perform integration testing between the
implemented components.

\hypertarget{list-of-definitions-and-abbreviations}{%
\subsection{List of definitions and
abbreviations}\label{list-of-definitions-and-abbreviations}}

\hypertarget{abbreviations}{%
\subsection{Abbreviations}\label{abbreviations}}

\begin{itemize}
\item
  \textbf{RASD} - Requirement Analysis and Specification Document
\item
  \textbf{DD} - Design Document
\item
  \textbf{JS} - JavaScript
\item
  \textbf{R\emph{n}} - Requirement \emph{n}
\item
  \textbf{IT\emph{n}} - Integration Test \emph{n}
\end{itemize}

\hypertarget{definitions}{%
\subsection{Definitions}\label{definitions}}

\begin{itemize}
\tightlist
\item
  \textbf{Client} - In this document client means a web app.
\end{itemize}

\hypertarget{reference-documents}{%
\subsection{Reference documents}\label{reference-documents}}

\begin{itemize}
\item
  Requirement Analysis and Specification Document (rasd.pdf)
\item
  Design Document (dd.pdf)
\end{itemize}

\hypertarget{document-structure}{%
\subsection{Document structure}\label{document-structure}}

\begin{itemize}
\item
  Chapter 1 presents the requirements that have been implemented in the
  prototype.
\item
  Chapter 2 presents the adopted programming languages and frameworks,
  justifying each choice.
\item
  Chapter 3 covers the structure of the source code.
\item
  Chapter 4 explains the testing process in greater detail.
\item
  Chapter 5 provides explanations on how to run, test and build the
  prototype.
\end{itemize}

\hypertarget{implemented-requirements}{%
\section{Implemented Requirements}\label{implemented-requirements}}

Requirements have the same nomenclature present in the RASD and DD.

We anticipate that there is no implementation in the prototype of the
\emph{Totem} described in the RASD and DD, as that would require extra
hardware to achieving pretty much the same functionality of the client
web app. Since the \emph{Totem} is identified as a client with special
privileges, therefore sharing the same endpoint of a normal client, it's
only a lack of hardware implementation on the client side.

\hypertarget{implemented}{%
\subsection{Implemented}\label{implemented}}

\begin{itemize}
\tightlist
\item
  R1 - Allow a User to sign up for an Account after providing a mobile
  phone number.
\item
  R2 - Allow a Registered User to find Stores nearby a specified
  location.
\item
  R4 - Allow a Registered User to get in the virtual line at a
  specifiedstore.
\item
  R6 - Allow a Registered User to preview an estimate of the queue time.
\item
  R8 - Allow a Registered User to cancel their reservation.
\item
  R9 - Allow a Registered User to leave the virtual queue.
\item
  R10 - Allow a Registered User \sout{and a Totem User} to retrieve a
  scannable QR Code/Barcode that they must present in order to be
  grantedaccess to a store. \emph{\textbf{Note::} We only implemented
  the part related to the Registered User, who can obtain a digital
  ticket containing the scannable code through the client web app.}
\item
  R12 - The System cancels User reservations in case of a major delay.
\item
  R14 - The System grants a User with a reservation access only within
  ashort time (set by the manager) after the User's time of reservation.
\item
  R15 - Allow System Managers to set the division of the maximum num-ber
  of people allowed between the normal queue, the priority queue for
  people with special needs and the book a visit slot ca-pacity.
\item
  R16 - The System calculates the average shopping time by recording
  every time a user enters and exits the store.
\item
  R17 - Allow System Managers to set a limit to the people allowed
  intothe store at a time.
\item
  R18 - Allow System Managers to choose the frequency and size of
  thetime slots.
\item
  R20 - Allow System Managers to know the current and past number
  ofpeople in the store.
\item
  R21 - Allow System Managers to check the current status of the
  queueand of the time slots.
\end{itemize}

\hypertarget{not-implemented}{%
\subsection{Not implemented}\label{not-implemented}}

\begin{itemize}
\item
  R3 - Allow a Registered User to filter out stores based on
  availabletimeframes, days and distance. \emph{\textbf{Explanation:}
  while the implementation isn't difficult, implementing this feature
  would have slowed down the development and testing process.}
\item
  R5 - Allow a Totem User to get in the virtual line of the store where
  thetotem is installed. \emph{\textbf{Explanation:} implementing the
  totem interface it's not really useful for a prototype, since this
  functionality works identically to a customer's client one.}
\item
  R11 - The System notifies the Users affected by delay.
  \emph{\textbf{Explanation:} it requires external paid service that
  isn't needed for a prototype, especially a web based one. Had we
  implemented a native mobile app, this feature would have been
  implemented.}
\item
  R13 - The System enforces the limits on the allowed number of
  con-current Customers inside a store by restricting the access at
  theentry points (for example, automatic doors or turnstile).
  \emph{\textbf{Explanation:} The Store API component is not implemented
  in the prototype as it requires custom integrations with existing API
  or hardware.}
\item
  R19 - Allow System Managers to know the average time spent in
  thestore. \emph{\textbf{Explanation:} Same reason of R13, it would
  require access to an existing store API/hardware.}
\end{itemize}

\hypertarget{adopted-technologies}{%
\section{Adopted Technologies}\label{adopted-technologies}}

The chosen frameworks and technologies have been chosen with regards the
Component View (Section 2.2) of the Design Document.

\hypertarget{frameworks-and-programming-languages}{%
\subsection{Frameworks and programming
languages}\label{frameworks-and-programming-languages}}

\hypertarget{back-end}{%
\subsubsection{Back-end}\label{back-end}}

We decided to implement the \emph{Application Server} with JavaScript
and \href{https://nodejs.org}{NodeJS}, because it offers great
horizontal scalability and generally good performance in handling
multiple simultaneous requests. In order to ease development, we chose
the industry standard server framework
\href{https://expressjs.com/}{ExpressJS}, which allows us to easily
implement a REST API for the clients.

The chosen DBMS is \href{https://www.mysql.com/}{MySQL}, because it's a
time-tested, fast and secure platform.

\hypertarget{front-end}{%
\subsubsection{Front-end}\label{front-end}}

In order to develop a working prototype in a short timeframe, we decided
to implement only the web application client, which offers the same
functionalities of a native application, exception made for push
notifications. The web app can be also easily ported into a
\href{https://developer.mozilla.org/en-US/docs/Web/Progressive_web_apps}{Progressive
Web App (PWA)}, which allows users to use it offline.

Of course, being the client a web application, the obvious choice
regarding suitable programming languages is JavaScript, allowing us to
have a codebase written in the same language.

We decided to use an industry-standard framework such as
\href{https://reactjs.org/}{React} to better organize the structure of
the web app, while maintaing high extendability, modularity and tidiness
of the code. In order to make requests against the REST API, we used the
new standard
\href{https://developer.mozilla.org/en-US/docs/Web/API/Fetch_API}{\texttt{fetch}
API}.

\hypertarget{additional-software}{%
\subsection{Additional software}\label{additional-software}}

In this part other notable tools are briefly explained.

\hypertarget{yarn}{%
\paragraph{Yarn}\label{yarn}}

In order to better organize the project dependencies, it's a wise choice
to use a packet manager. By having a codebase written entirely in JS, we
adopted \href{https://yarnpkg.com/}{Yarn}, a fast and reliable packet
manager for JavaScript. It also handles custom scripts to ease the
development workflow.

\hypertarget{jest}{%
\paragraph{Jest}\label{jest}}

To perform unit and integration testing, we've chosen
\href{https://jestjs.io/}{Jest}, an industry-standard JavaScript testing
framework.

\hypertarget{docker}{%
\paragraph{Docker}\label{docker}}

To easily deploy a configured database for development, we utilized a
MySQL image with \href{https://www.docker.com/}{Docker}, an
industry-standard container solution.

\hypertarget{source-code-structure}{%
\section{Source code Structure}\label{source-code-structure}}

TODO

\hypertarget{testing}{%
\section{Testing}\label{testing}}

TODO

(Refer to DD + procedure, explain at least system tests, would be good
to consider integration tests)

\hypertarget{installation-instructions}{%
\section{Installation Instructions}\label{installation-instructions}}

Here we report the instructions to install all software dependencies
needed to run the service. This instructions are also available in the
GitHub repository in the ITD folder.

\hypertarget{installing-the-toolchain}{%
\subsection{Installing the toolchain}\label{installing-the-toolchain}}

\begin{itemize}
\tightlist
\item
  Install \href{https://nodejs.org/}{NodeJS}, possibly the LTS version
  (14.15.4).

  \begin{itemize}
  \tightlist
  \item
    If you use tools such \texttt{nvm}, please run
    \texttt{nvm\ install\ -\/-lts} and \texttt{nvm\ use\ -\/-lts}
  \end{itemize}
\item
  Open the terminal and run \texttt{npm\ install\ -g\ yarn} to install
  the packet manager we're using, \href{https://yarnpkg.com}{Yarn}
\item
  If you are developing this project, installing
  \href{https://github.com/remy/nodemon}{Nodemon} is recommended
  \texttt{npm\ install\ -g\ nodemon}. Please note that it's already
  included in the \emph{developer dependencies}, so if you do not wish
  to install it globally, skip this step.
\item
  Clone the repository using either one of the following commands

  \begin{itemize}
  \tightlist
  \item
    \texttt{git\ clone\ https://github.com/ian-ofgod/DiDioLavoreFuscoFranchini-CLup}
  \item
    \texttt{gh\ repo\ clone\ ian-ofgod/DiDioLavoreFuscoFranchini-CLup}
  \end{itemize}
\item
  Change directory into the ITD folder
  \texttt{cd\ DiDioLavoreFuscoFranchini-CLup/ITD}
\item
  Run \texttt{yarn\ install} to install the dependencies
\end{itemize}

\hypertarget{setting-up-the-database}{%
\subsection{Setting up the database}\label{setting-up-the-database}}

You'll have to install \href{https://www.docker.com/}{Docker} on your
device, and make sure
\href{https://docs.docker.com/compose/install/}{docker-compose} is
installed as well.

\begin{itemize}
\tightlist
\item
  Run \texttt{docker-compose\ up\ -d} to spin up the images specified in
  the \texttt{docker-compose.yml} file.
\end{itemize}

Visit \url{http://localhost:8081} to see if
\href{https://www.adminer.org}{Adminer} is running, and log in with the
credentials specified in the \texttt{.env} file (server: \texttt{db},
username: \texttt{clup\_admin}, password: \texttt{clup}, database:
\texttt{db\_clup}).

\begin{quote}
\hypertarget{alternative-without-docker}{%
\subsubsection{Alternative without
Docker}\label{alternative-without-docker}}

\begin{itemize}
\tightlist
\item
  Install \href{https://www.mysql.com}{MySQL} server, then create a
  schema and a user with the credentials found in the \texttt{.env}
  file, otherwise create your own and edit the \texttt{.env} file.
\end{itemize}
\end{quote}

\begin{itemize}
\tightlist
\item
  In case of errors about permissions, try running the following
  command, especially if you are not using Docker:
\end{itemize}

\begin{Shaded}
\begin{Highlighting}[]
\KeywordTok{create} \FunctionTok{user}\NormalTok{ clup\_admin@localhost }\KeywordTok{identified} \KeywordTok{with}\NormalTok{ mysql\_native\_password }\KeywordTok{by} \StringTok{\textquotesingle{}clup\textquotesingle{}}\NormalTok{;}
\KeywordTok{grant} \KeywordTok{all} \KeywordTok{privileges} \KeywordTok{on} \OperatorTok{*}\NormalTok{ . }\OperatorTok{*} \KeywordTok{to}\NormalTok{ clup\_admin@localhost;}
\end{Highlighting}
\end{Shaded}

\hypertarget{verify-the-installation}{%
\subsection{Verify the installation}\label{verify-the-installation}}

\begin{itemize}
\tightlist
\item
  Run \texttt{yarn\ server-dev} to run the developer server
\item
  Run \texttt{yarn\ client-dev} to run the client server
\item
  Run \texttt{yarn\ client-build} to compile the client package (output
  is in \texttt{dist})
\item
  Run \texttt{yarn\ test} to run the full test suite with
  \href{https://jestjs.io}{Jest}
\end{itemize}

\end{document}
