\section{Introduction}\label{sec:intro}

General introduction. \todo{This is a todo}Text.

General introduction. \warning{This is a warning}Text.

General introduction. \request{This is a request}Text.

General introduction. \info{This is an info}Text.

\subsection{Purpose}
This document is the Requirement Analysis and Specification Document for the Customers Line-Up system.
The purpose of this document is to describe the system focusing on scenarios, use cases, requirements and specifications,
analyzing what the software will do, how it will be used and the constraints under which it will operate.
This document is intended both for users and developers.

\subsection{Scope}
\info{here we include an analysis of the world and of the shared phenomena}

\emph{Customers Line-Up} is a tool that allows managers to regulate the influx of people in physical stores and offers
reduce the time spent by customers waiting in line, especially in emergency lockdown scenarios.
The goal of the system is to help businesses abide by the regulation imposing limits on the number of people
who can visit stores at a time and prohibiting the formation of crowds such as queues.
This tool reaches the goal by offering a number of functionalities, including:\todo{revise/add more}
\begin{itemize}
    \item access to the service via mobile app or website
    \item physical alternatives for people that do not have Internet access
    \item monitoring the amount of people in a store
    \item booking a visit, notifying customers of any change in the schedule
    \item suggesting alternate stores and/or time frames
    \item tracking the time spent by customers to estimate waiting times
\end{itemize}

\subsection{Definitions, Acronyms, Abbreviations}
\subsection{Revision History}
\subsection{Reference History}
\subsection{Document Structure}
